\documentclass[12pt,a4]{article}

\setlength{\textwidth}{17cm}
\setlength{\textheight}{24cm}
\setlength{\leftmargin}{-1cm}
\setlength{\topmargin}{-2cm}
\setlength{\oddsidemargin}{0cm}
\setlength{\evensidemargin}{0cm}

\usepackage{xltxtra}
\setmainfont{IPAPMincho}
\setsansfont{IPAPGothic}
\setmonofont{IPAGothic}
\XeTeXlinebreaklocale "ja"

\usepackage{hyperref}
\usepackage{listings}
\usepackage{verbatim}
\usepackage{amsmath}

\title{物理学情報処理論1レポート問題1}
\author{nona7}
\date{}

\begin{document}
\maketitle

\section{物理量の次元}
\subsection{次の物理量の次元を求めよ。ただし、長さ、質量、時間、電流、温度の次元をL,M,T,A, $\Theta$ とする。}
\subsubsection{運動量}
[運動量] = [ $ \frac{ML}{T} $ ]
\subsubsection{エネルギー}
[エネルギー] = [ $ \frac{ML^2}{T^2} $ ]
\subsubsection{角速度}
[角速度] = [ $ \frac{1}{T} $ ]
\subsubsection{角運動量}
[角運動量] = [ $ \frac{ML^2}{T} $ ]
\subsubsection{振動数}
[振動数] = [ $ \frac{1}{T} $ ]
\subsubsection{力積}
[力積] = [ $ \frac{ML}{T} $ ]
\subsubsection{電荷}
[電荷] = [ $ AT $ ]
\subsubsection{電場}
[電場] = [ $ \frac{ML}{AT^3} $ ]
\subsubsection{磁束密度}
[磁束密度] = [ $ \frac{M}{AT^2} $ ]
\subsubsection{抵抗}
[抵抗] = [ $ \frac{ML^2}{A^2T^3} $ ]
\subsubsection{圧力}
[圧力] = [ $ \frac{M}{LT^2} $ ]
\subsubsection{比熱}
[比熱] = [ $ \frac{L^2}{T^2\Theta} $ ]
\subsubsection{エントロピー}
[エントロピー] = [ $ \frac{ML^2}{T^3} $ ]

\section{無次元化}
\subsection{減衰振動の方程式と無次元化}
\subsubsection{(a)}
$ T = \sqrt{\frac{m}{k}} $ として
$ t = T t^*, x = Lx^* $ とおく。
$ dt = T dt^*, dx = L dx^* $ である。
$ m \frac{d^2x}{dt^2} = -\nu v - kx $ に代入し、
$ m \frac{d^2(Lx^*)}{T^2 {dt^*}^2} = -\nu \frac{d Lx^*}{T dt^*} - k(Lx^*) $ を得る。整理して、
$ \frac{d^2x^*}{{dt^*}^2} = - \frac{T\nu}{m} \frac{dx^*}{dt^*} -x^* $。よって
$ \frac{d^2x^*}{{dt^*}^2} = - 2 \alpha \frac{dx^*}{dt^*} - x^* $ の $ \alpha $ は
$ \alpha = \frac{\nu}{2\sqrt{mk}} $ である。

\subsubsection{(b)}
$ \frac{d^2x}{dt^2} = - 2 \alpha \frac{dx}{dt} - x $ , $ x(0) = 1 $ , $ \frac{dx}{dt}(0) = 0 $ \\
の特性方程式 $ y^2 + 2\alpha + 1 = 0 $ を解くと、
$ y = -\alpha \pm \sqrt{\alpha^2-1} $

\medskip
\noindent
$ \alpha \pm 1 $ の時 \\
$ x = e^{\mp t}(C_1 + C_2 t) $ \\
$ x(0) = 1 $ より $ C_1 = 1 $ \\
$ \frac{dx}{dt}(0) = 0 $ より $ C_2 = -1 $ \\
よって \\
$ x = e^{\mp t}(1 - t) $ (複号同順)

\medskip
\noindent
$ | \alpha | > 1 $ の時 \\
$ x = e^{(-\alpha+\sqrt{\alpha^2-1})t} C_1 + e^{(-\alpha-\sqrt{\alpha^2-1})t} C_2 $\\
$ x(0) = 1 $ より $ C_1 + C_2 = 1 $ \\
$ \frac{dx}{dt}(0) = 0 $ より $ C_1 - C_2 = \frac{\alpha}{\sqrt{\alpha^2-1}} $ \\
合せて \\
\[
\begin{cases}
  C_1 = \frac{1}{2}(1+\frac{\alpha}{\sqrt{\alpha^2-1}}) \\
  C_2 = \frac{1}{2}(1-\frac{\alpha}{\sqrt{\alpha^2-1}}) 
\end{cases}
\]
となり、
$ x = \frac{1}{2} (e^{(-\alpha+\sqrt{\alpha^2-1})t} (1+\frac{\alpha}{\sqrt{\alpha^2-1}}) + e^{(-\alpha-\sqrt{\alpha^2-1})t} (1-\frac{\alpha}{\sqrt{\alpha^2-1}})) $\\

\medskip
\noindent
$ | \alpha | < 1 $ の時 \\
$ x = e^{-\alpha t}(C_1 \cos{\sqrt{1-\alpha^2}t} + C_2 \sin{\sqrt{1-\alpha^2}t}) $\\
$ x(0) = 1 $ より $ C_1 = 1 $ \\
$ \frac{dx}{dt}(0) = 0 $ より $ C_2 = 0 $ \\
よって \\
$ x = e^{-\alpha t} \cos{\sqrt{1-\alpha^2}t}  $

\end{document}
